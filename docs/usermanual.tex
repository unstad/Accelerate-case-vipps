\documentclass[11pt,a4paper,english]{article}
\usepackage[english]{babel}
\usepackage[utf8]{inputenc}
\usepackage{amsmath,amsfonts,amssymb,graphicx,textcomp,caption,color,listings,subcaption}
\usepackage[T1]{fontenc}
\usepackage[section]{placeins}
\usepackage[hidelinks,bookmarksnumbered]{hyperref}
\usepackage[all]{hypcap}
\usepackage[top=1in]{geometry}
\renewcommand{\thesubsection}{\thesection.\alph{subsection}}

\title{Netbutikk documentation}
\author{Kristian Nilsen, Jon Erik Ullvang, Vinh Laph Nguyen,\\ Katharina Unstad, Espen Sotnakk}

\begin{document}
\maketitle
\fontfamily{ptm}\selectfont
\section*{Home}
This is the shop homepage where the products are shown as shown in Figure~\ref{fig:home}. Pressing on an item will open a modal showing more information and an button to add the item to cart.
\begin{figure}[htbp]
  \centering
  \includegraphics[scale=0.2]{home}
  \caption{Shop homepage showing products in a grid.}
  \label{fig:home}
\end{figure}
\section*{Product}
This modal opens when the product is pressed as shown in Figure~\ref{fig:item}. The modal shows more information about the item and a button to add item to cart. Pressing the x in the top left corner or pressing outside of the modal closes modal, and the home page is shown again.
\begin{figure}[htbp]
  \centering
  \includegraphics[scale=0.2]{item}
  \caption{Modal showing the product information.}
  \label{fig:item}
\end{figure}
\section*{Cart}
The cart page is shown in Figure~\ref{fig:cart} and shows a list of the items added to cart. Each item has a button to remove the item from cart. A sum of the amount for the items added to cart is shown with a button to continue to order page.
\begin{figure}[htbp]
  \centering
  \includegraphics[scale=0.2]{cart}
  \caption{}
  \label{fig:cart}
\end{figure}
\section*{Order details}
This is the page where the users fills in information for placing an order. All fields are required with valid data.
\begin{figure}[htbp]
  \centering
  \includegraphics[scale=0.2]{order}
  \caption{Shows the form required for making an order.}
  \label{fig:order}
\end{figure}
\section*{Register}
Register page is shown in Figure~\ref{fig:register}. There are two options for registering a new account. One is to create user with email and password. The other option is to use facebook. Account confirmation is required after registering a new user.
\begin{figure}[htbp]
  \centering
  \includegraphics[scale=0.3]{register}
  \caption{Register page.}
  \label{fig:register}
\end{figure}
\section*{Login}
The login page is shown in Figure~\ref{fig:login}. The user has two options to login. The first is with an account the user created. The other is with authenticate with facebook.
\begin{figure}[htbp]
  \centering
  \includegraphics[scale=0.3]{login}
  \caption{Login page.}
  \label{fig:login}
\end{figure}
\section*{Order history}
\begin{figure}[htbp]
  \centering
  \includegraphics[scale=0.3]{history}
  \caption{Order history page.}
  \label{fig:history}
\end{figure}
\end{document}
